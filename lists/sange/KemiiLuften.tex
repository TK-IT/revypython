\begin{Sang}[Der er Noget i Luften]{Kemi i Luften}

%TID: 1 min 30 sek

\begin{Persongalleri}
\item Sanger$_A$ A
\item Sanger$_B$ B
\item Sanger$_C$ C
\item Sanger$_D$ D
\end{Persongalleri}

\begin{Regi}
De fire sangere er bange for kemien omkring dem, og fortæller hinanden om farer og gode råd
\end{Regi}

\begin{Strofe}
  \Vers[A]{Der er noget i luften}
  \Vers[---]{jeg ved ikke hvad}
  \Vers[B]{De skrev det er farligt}
  \Vers[---]{i mit dameblad}
  \Vers[C]{Når ungerne hyler}
  \Vers[---]{er det ethanyler}
  \Vers[---]{der gør dem så ked’}
  \Vers[D]{Det er molekyler i maden}
  \Vers[---]{der gør mig så fed}
\end{Strofe}

\begin{Strofe}
  \Vers[B]{Der er noget i plastik}
  \Vers[---]{som gør det så nem’}
  \Vers[---]{at støbe den i samme}
  \Vers[---]{form som et lem}
  \Vers[A-D]{\regi{Henvendt til publikum}Når dildoen fryder}
  \Vers[---]{så husk det betyder}
  \Vers[---]{kemi på din hud}
  \Vers[C]{\regi{Igen til de andre sangere}Jeg har endda hørt at det gir’}
  \Vers[---]{vitaminunderskud}
\end{Strofe}

\begin{Strofe}
  \Vers[D]{Der er noget i sæben}
  \Vers[---]{som skummer så flot}
  \Vers[A]{Det er for at gøre}
  \Vers[---]{dig syg - et komplot!}
  \Vers[A-D]{\regi{Henvendt til publikum}Vi er alle ofre}
  \Vers[---]{for kemiske stoffer}
  \Vers[---]{som ilt og oxygen}
  \Vers[---]{Ak, hvem der ku’ undgå kemi}
  \Vers[---]{bare leve helt rent}
\end{Strofe}

%Forslag (sidste linje):
% li'som Rolf - han er ren

%Forslag (de to sidste linjer):
% Ja vælg mat-fysik hvis du vil
% Være Rolf som en ren

%Forslag (Diana - 3 sidste linjer):
% Som ilt der giver fnat
% ak hvem der ku' undgå kemi
% bare læse fys/mat

\end{Sang}
\begin{Footer}
Hans Gert Christensen
\end{Footer}
